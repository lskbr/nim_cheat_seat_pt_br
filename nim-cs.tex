\documentclass[10pt, a4paper]{article}

% imports
\usepackage[utf8]{inputenc}
\usepackage[landscape, margin=2px]{geometry}
\usepackage[default]{raleway}
\usepackage{hyperref}
\usepackage{tcolorbox}
\usepackage{multicol}

% colors
\definecolor{nimyellow}{HTML}{FFE953}
\definecolor{nimgray}{HTML}{C5C1B9}
\definecolor{nimlightgrey}{HTML}{2B2E3B}
\definecolor{nimdarkgrey}{HTML}{171921}

% init libs settings
\colorlet{xx}{nimlightgrey}
\tcbuselibrary{most, listingsutf8, minted}
\tcbset{tcbox width=auto, left=1pt, top=0pt, bottom=0pt,
right=1pt, boxsep=0pt, middle=1px}


%\usemintedstyle[dark]

\setlength{\columnsep}{0pt}
\setlength{\parindent}{0pt}
\setlength{\parskip}{0pt plus 0pt}
\title{Nim Cheatsheet}
\author{Juan Carlos}
\setlength{\columnseprule}{0pt}
\setlength{\premulticols}{0pt}
\setlength{\postmulticols}{0pt}
\setlength{\multicolsep}{0pt}
\setlength{\columnsep}{0pt}
\pagestyle{empty}
\pagecolor{nimlightgrey}
\color{white}


% codebox function
\newtcblisting{codebox}[2]{
  colback=nimgray,
  colframe=nimyellow,
  listing only,
  minted options={style=emacs, fontsize=\small, breaklines, autogobble, numbersep=0pt},
  left=0pt,
  top=0pt,
  bottom=0pt,
  enhanced,
  title=#2,
  coltitle=black,
  fonttitle=\bfseries,
  listing engine=minted,
  minted language=#1
}

\begin{document}

\begin{multicols}{4}
\href{https://nim-lang.org}{
  \color{nimyellow}\textbf{Nim Cheatsheet}} \href{https://t.me/NimArgentina}{\color{nimgray}\emph{criado por NimArgentina} traduzido port \href{https://t.me/nimbrasil}{Nim Brasil}
}


\begin{codebox}{nim}{Alô Mundo}
  echo "Alô Mundo!"
\end{codebox}


\begin{codebox}{nim}{Tipos}
  assert 42 is int           # Inteiro
  assert true is bool        # Booleano
  assert "42" is string      # String
  assert 42.0 is float       # Float
  assert @[1, 2, 3] is seq   # Sequencia
  assert  (1, 2, 3) is tuple # Tupla
  assert  {1, 2, 3} is set   # Conjunto/Set
  assert %*{"key":"val"} is JsonNode # JSON
\end{codebox}


\begin{codebox}{nim}{Variáveis / Constantes}
  var numero = "Mutável"
  let numero = "Imutável"
  const numero = "Imutável" # em Tempo de compilação
\end{codebox}


\begin{codebox}{nim}{Matemática}
  echo 1 + 2, 2 - 1 # Soma e subtração
  echo 2 * 2, 4 / 2
  echo 4 mod 2, 8 xor 2 # Módulo & Xor
  echo 8 shl 2, 8 shr 2 # Shift Left & Right
\end{codebox}


\begin{codebox}{nim}{Iteradores}
for i in 0..5: echo i  # 0, 1, 2, 3, 4, 5
for i in 'a'..'e': echo i # a, b, c, d, e
\end{codebox}


\begin{codebox}{nim}{Operador ternário / if ... else}
if conditional: "valor_0" else: "valor_1"
\end{codebox}


\begin{codebox}{nim}{Case Switch}
case "gato"
of "cachorro": echo "Cachorro"
of "gato": echo "Gato"
else:     echo "Outro"
\end{codebox}


\begin{codebox}{nim}{Fatias / Slices}
let variable = [1, 2, 3, 4]
assert variable[1..2] == @[2, 3]
assert variable[0..3] == @[1, 2, 3, 4]
assert variable[1..^1] == @[2, 3, 4]
assert variable[0..^2] == @[1, 2, 3]
\end{codebox}


\begin{codebox}{nim}{Identação opcional: Expressões em linha}
let a = try: 1+2 except: 9 finally: echo 0
\end{codebox}


\begin{codebox}{nim}{Imports}
  import os, json, strutils, othermodule
  include "folder/file.nim", "owo.nimf"
\end{codebox}


\begin{codebox}{nim}{Operações com String}
import strutils, editdistance, base64
echo "con" & "cat"
echo " ABC ".strip(), "A_B_C".normalize()
echo "OwO".countLines(),"UwU".splitLines()
echo "a,b,c".split(","), "a".repeat(9)
echo parseInt("42"), parseFloat("3.1")
echo "a".indent(9), " a".unindent(1)
echo "abc".startswith("a") # endswith("c")
echo ["a", "b", "c"].join(";")
assert encode("Nim") == "Tmlt" # base64
assert editDistance("Cats", "Bats") == 1
\end{codebox}


\begin{codebox}{nim}{Literais string com formatação}
import strformat
echo fmt"Foo {1 + 2} Bar {variable} Baz"
\end{codebox}


\begin{codebox}{nim}{Funções}
proc name(argument: int): int =
  result = argument + 2
proc fibo(n: int): int =
  if n <= 2: 1 else: fibo(n-1) + fibo(n-2)
\end{codebox}


\begin{codebox}{nim}{Funções Anônimas (Lambda)}
var fn = (proc (a, b: int): int = a + b)
\end{codebox}


\begin{codebox}{nim}{Funções Arrow}
import sugar
let fn = (a, b: int) => a + b
\end{codebox}


\begin{codebox}{nim}{UFCS}
echo("it is")     # echo "it is"
"the same".echo() # "the same".echo
\end{codebox}


\begin{codebox}{nim}{E/S: Leitura e escrita de arquivos}
write_file("arquivo.txt", "AlÔ")
assert read_file("arquivo.txt") == "Alô"
const txts = static_read("arquivo.txt")
\end{codebox}


\begin{codebox}{nim}{E/S: Leitura de arquivos linha a linha}
for linha in lines("log.txt"): echo linha
\end{codebox}


\begin{codebox}{nim}{E/S: Lista pastas e arquivos recursivamente}
import os
for fl in walkPattern("**/*.txt"): echo fl
for tipo,fl in walkDirRec("/tmp"): echo fl
\end{codebox}


\begin{codebox}{nim}{Map / Filter / Zip / Deduplicate}
import sequtils
proc isPositive(arg: int): bool = arg > 0
echo map([1, 2, -3, 5, -9], isPositive)
echo filter([1, 2, -3, 5, -9], isPositive)
echo zip(@[1, 2, -3], @[5, -9, 42])
assert deduplicate(@[1,1,2,3]) == @[1,2,3]
\end{codebox}


\begin{codebox}{nim}{Tabelas (Mapeamento com dicionário)}
import tables
var dictionary = {"a": 1, "b": 2}.toTable
assert dictionary["a"] == 1
dictionary["a"] = 42
for key, val in dictionary: echo key, val
\end{codebox}


\begin{codebox}{nim}{Sobrecarga de Operadores}
template `:=`(a, b): untyped = (var a=b; a)
assert foo := 42 == 42  # Operador Walrus
\end{codebox}


\begin{codebox}{nim}{Terminal colorido}
import terminal
styledEcho(bgBlue, fgRed, "Color echo")
\end{codebox}


\begin{codebox}{nim}{HTTP: GET / POST / Headers}
import httpclient
const url = "http://example.io" # Example
var hdr = newHttpHeaders({"DNT": "1"})
var cliente = newHttpClient(headers=hdr)
var resp0 = cliente.get(url)
var resp1 = cliente.post(url, "data")
\end{codebox}


\begin{codebox}{nim}{Processos do SO (Processos Nativos)}
import osproc
let (output, errorCode) = execCmdEx("pwd")
\end{codebox}


\begin{codebox}{nim}{Tempo: DateTime / Parse / Format}
import times # monotimes for Monotonic
echo now() - 2.days + 9.months + 1.years
let dt = parse("2000-01-01", "yyyy-MM-dd")
echo dt.format("yyyy-MM-dd")
\end{codebox}


\begin{codebox}{nim}{Objetos: Básico}
  type Gato = object ## Documentacao
    idade: int ## Documentacao
  proc miau(this: Gato) = echo "Miau"
  Gato(idade: 5).miau() # Instância
\end{codebox}


\begin{codebox}{nim}{Objetos: OOP / Herança}
type Animal = ref object of RootObj # Base
  idade: int # Atributos
type Gato = ref object of Animal
  nome: string  # Atributos
proc incrementa_idade(self: Gato) =
  self.idade.inc()
var gatinho = Gato(nome: "Tom", idade: 3)
gatinho.increment_age()
\end{codebox}


\begin{codebox}{nim}{Criação de nomes de identificadores}
template createFunction(arg) =
  proc `arg  World`(): string = "Exemplo"
createFunction(hello)
assert helloWorld() == "Exemplo"
\end{codebox}


\begin{codebox}{nim}{Testes unitários: Setup / TearDown / Suite}
import unittest
suite "Nome do Teste":
  setup:
    echo "Teste Setup"
  teardown:
    echo "Teste Teardown"
  test "example":
    assert 42 == 42, "Error Message"
\end{codebox}


\begin{codebox}{nim}{Seq Comprehensions}
import sugar
let variable = collect(newSeq):
  for item in @[-9, 1, 42, 0, -1, 9]: item
\end{codebox}


\begin{codebox}{nim}{Table Comprehensions}
import sugar
let variable = collect(initTable(4)):
  for key,val in @[-9,42,0,9]: {key:val}
\end{codebox}


\begin{codebox}{nim}{RegEx: Expressões Regulares (PCRE)}
import re
echo "a=1;b=2".replace(re"(\w+)=(\w+)")
for s in "abcd".findAll(re"bcd"): echo s
if "key=42" =~ re"(\w+)\=(\w+)":
  assert matches[0] == "key"
  assert matches[1] == "42"
\end{codebox}


\begin{codebox}{nim}{PEG: Parsing Expression Grammar}
import pegs # PEG é uma alternativa a RegEx
if "key=42" =~ peg"{\w+}'='{\w+}":
  assert matches[0] == "key"
  assert matches[1] == "42"
\end{codebox}


\begin{codebox}{nim}{Macros: Básico}
import macros
macro gera_funcao(a, b: int): typed =
  quote do:
    proc examplo(): int = `a` + `b`
gera_funcao(6, 4)
assert examplo() == 10
\end{codebox}


\begin{codebox}{nim}{Avançado: FFI C / C++ / JS / ObjC}
proc printf(s: cstring) {.
  importc, header: "stdio.h".}
printf("C/C++ function on Nim!".cstring)
\end{codebox}


\begin{codebox}{nim}{Avançado: Gerenciamento de memória manual}
let variable = create(int, sizeOf(int))
variable[] = 42  # Assign value 42
assert variable[] == 42
zeroMem(variable, sizeOf(variable.type))
\end{codebox}


\begin{codebox}{nim}{HTTP: Servidor Web Multi-Thread Async}
import jester
routes: get "/url": resp "Hello World"
\end{codebox}


\begin{codebox}{nim}{Avançado: L-Value por Referêcia}
import decls
var exemplo = @[1, 2, 3]
var varXref1 {.byaddr.} = exemplo[0]
var varXref2 {.byaddr.} = exemplo[0]
assert varXref1.addr == varXref2.addr
assert varXref1 is int and varXref1 is int
varXref1 += 42
\end{codebox}


\begin{codebox}{nim}{Blocos With (Gerenciamento de Contexto)}
import with
var variable = 44
with variable:
  += 4
  -= 5

assert variable == 43
\end{codebox}


\begin{codebox}{nim}{Mimetypes}
import mimetypes
var m = newMimetypes()
assert m.getMimetype("html") == "text/html"
assert m.getExt("text/html") == "html"
assert m.getMimetype("") == "text/plain"
m.register(ext = "customExtension", mimetype = "text/customMime")
\end{codebox}


\begin{codebox}{nim}{SHA1 / MD5}
import md5
assert toMD5("abc") is MD5Digest
assert getMD5("abc") == "900150983cd24fb0d6963f7d28e17f72"

assert secureHash("John Doe") is SecureHash
assert secureHashFile("file.txt") is SecureHash
\end{codebox}


\begin{codebox}{nim}{URL / Browser}
import browsers
openDefaultBrowser("https://nim-lang.org")
\end{codebox}


\begin{codebox}{nim}{Linux Distros}
import distros
if detectOs(Ubuntu): echo "Ubuntu Distro"
\end{codebox}


\begin{codebox}{nim}{CPU Cores}
import cpuinfo
echo countProcessors()
\end{codebox}


\begin{codebox}{nim}{Configurações do Compilador}
import compilesettings
let x = SingleValueSetting.nimcacheDir
let nimcache = querySetting(x)
let z = MultipleValueSetting.nimblePaths
let nimblePaths = compileSettingSeq(z)
\end{codebox}


\begin{codebox}{nim}{JavaScript Console}
import jsconsole
console.log("echo on web browser console!")
\end{codebox}


\begin{codebox}{nim}{Bloco: Novo Contexto Explícito Nomeado}
block nome_do_bloco: # Nome Opcional
  let scopedVariable = "A new Scope"
  proc funcion() = echo 42
  # break nome_do_bloco # Saída do bloco

# echo scopedVariable
# Error: undeclared: "scopedVariable"
# funcion()
# Error: undeclared: "funcion"
\end{codebox}


\begin{codebox}{nim}{Enums}
import strutils
type Fruta = enum Maca, Banana, Kiwi
let fruto = parseEnum[Fruta]("banana")
assert Maca is Fruta
\end{codebox}


\begin{codebox}{nim}{Variáveis de Ambiente (Env Vars)}
import os
echo getEnv("HOME") # Ler
setEnv("Clave", "Valor") # Escrever
\end{codebox}


\begin{codebox}{nim}{Executar ao Sair}
proc funcionASair() {.noconv.} =
  echo "Esta se executa ao terminar"

addQuitProc(funcionAlSalir)
\end{codebox}


\begin{codebox}{nim}{Executar com CTRL+C}
proc funcionACtrlC() {.noconv.} =
  echo "Esta se executa com CTRL+C"

setControlCHook(funcionACtrlC)
\end{codebox}


\begin{codebox}{nim}{Exemplos Executáveis Auto-Documentação}
runnableExamples: # nim doc file.nim
  echo "Self-Documentation"
\end{codebox}


\begin{codebox}{nim}{Logging}
import logging
addHandler(newRollingFileLogger("app.log"))
info("Log em Arquivo com formatação e rotação automática")
\end{codebox}


\begin{codebox}{nim}{Conceitos}
type CanMove = concept x
  move(x)
type Person = object

proc move(this: Person) = echo "move!"
proc walk(this: CanMove) = move(this)

let persona = Person()
walk(persona)
\end{codebox}


\begin{codebox}{nim}{Email}
import smtp
var mensagem = createMessage("Titulo", "Mensagem",  @["tu@mail.io"])
let smt = newSmtp(useSsl=true, debug=true)
smt.connect("smtp.gmail.com", Port(465))
smt.auth("username", "password")
smt.sendmail("destino@server.io", @["tu@mail.io"], mensagem)
\end{codebox}


\begin{codebox}{nim}{MemFiles: Arquivo Mapeado em Memória}
import memfiles
let memo = memfiles.open("arquivo.md",fmReadWrite)
let ponteiro = memo.mem()
let conteudo = cast[cstring](ponteiro)
echo conteudo
\end{codebox}


\begin{codebox}{nim}{Markdown/RST em HTML}
import packages/docutils/rstgen, strtabs
echo rstToHtml("*Olá* **Mundo**", {}, newStringTable(modeStyleInsensitive))
\end{codebox}


\begin{codebox}{nim}{Diff}
import experimental/diff
let txt0 = """Linha 1
linha2"""
let txt1 = """linha 1
Linha alterada!"""

echo diffText(txt0, txt1)
echo diffInt([0, 1, 2, 3], [-1, 1, 2, 3])
\end{codebox}


\begin{codebox}{nim}{Gerador de HTML}
import htmlgen # SVG e MathML suportados
echo h1(a(href="nim-lang.org", "Nim"))
assert form(action="test")
\end{codebox}


\begin{codebox}{nim}{Genéricos}
func min[T](x, y: T): auto =
  if x < y: x else: y

echo min(2, 3) # int
echo min("foo", "bar") # string
\end{codebox}


\begin{codebox}{nim}{Expressão de Bloco}
let variable = block:
  var fib = @[0, 1]
  for i in 0..10:
    fib.add fib[^1] + fib[^2]
  fib
\end{codebox}


\begin{codebox}{nim}{Iteradores}
iterator items(a: string): char =
  var i = 0
  while i < len(a):
    yield a[i]
    inc i

for item in items("Alô mundo"): echo item
\end{codebox}


\begin{codebox}{nim}{Exceções Personalizadas}
raise newException(IOError, "Mensagem")
raise newException(Exception, "Mensagem")
\end{codebox}




\end{multicols}
\end{document}
